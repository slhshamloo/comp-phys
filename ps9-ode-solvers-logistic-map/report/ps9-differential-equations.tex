\documentclass[12pt,a4paper]{article}
\usepackage{algorithm, algpseudocode, amsmath, amssymb, amsthm, csquotes, empheq, geometry, graphicx, hyperref, listings, multirow, siunitx, subcaption, upgreek}
\usepackage[italicdiff]{physics}
\usepackage[section]{placeins}
\usepackage[justification=centering]{caption}

\title{Computational Physics\\Problem Set 9}
\author{Saleh Shamloo Ahmadi\\Student Number: 98100872}
\date{December 12, 2021}

\hypersetup{colorlinks=true, urlcolor=cyan}
\newcommand{\fig}{../fig}
\newcommand{\fighere}[2]{\begin{figure}[htb!]
    \centering
    \includegraphics[width=#2\linewidth]{\fig/#1}
\end{figure}}
\newcommand{\multlinecell}[1]{\begin{tabular}[c]{@{}c@{}}#1\end{tabular}}
\newcommand*{\defeq}{\mathrel{\vcenter{\baselineskip0.5ex \lineskiplimit0pt
			\hbox{\scriptsize.}\hbox{\scriptsize.}}}
			=}

\begin{document}
	\maketitle
    \section{RC Circuit: First Order Ordinary Differential Equation}
    In a RC circuit where a battery with voltage $V_0$ is connected to a capacitor and a resistor (kvl)
    \begin{equation}
        V_B + V_C + V_R = 0 \implies V_0 - \frac{q}{C} - IR = 0.
    \end{equation}
    Here, the current charges the capacitor, so $I = \dot{q}$ and therefore
    \begin{equation}
        \dot{q} = \frac{V_0}{R} - \frac{q}{RC}.
    \end{equation}
    Defining $q_0 \defeq CV_0$ and assuming $q(0) = 0$
    \begin{equation}
        q(t) = q_0(1 - e^{-t/RC}).
    \end{equation}
    
    Here, the Euler method and the classic Runge-Kutta methods are compared against the exact solution.

    Also, in the last figure, it is shown that the method $x_{n+1} = x_{n-1} + 2\Delta{t}\dot{x}(t_n, x_n)$
    is unstable.

    \newgeometry{top=0.1in, bottom=0.1in}
    \thispagestyle{empty}
    \fighere{rc-sol}{0.86}
    \fighere{rc-error}{1}
    \fighere{rc-unstable}{0.86}
    \restoregeometry
    \section{Simple Harmonic Oscillator:\\Second Order Ordinary Differential Equation}
    For the simple harmonic oscillator
    \begin{gather}
        \ddot{x} = -\omega x, \\
        \text{for $v(0) = \dot{x}(0) = 0$}
        \left\{\begin{aligned}
            x(t) &= x_0\cos(\omega t) \\
            v(t) &= x_0\omega\sin(\omega t)
        \end{aligned}\right.
    \end{gather}

    Here, 6 numerical methods are compared agains the exact solution:
    \begin{table}[hbt!]
        \centering
        \begin{tabular}{|c|c|c|c|}
            \hline
            Method & \multlinecell{Local\\Error} & \multlinecell{Global\\Error} & \multlinecell{Global Velocity\\($\dot{x}(t)$) Error} \\
            \hline
            Euler Method & $\mathcal{O}(\Delta{t}^2)$ & $\mathcal{O}(\Delta{t})$ & $\mathcal{O}(\Delta{t})$ \\
            \hline
            \multlinecell{Euler--Cromer Method\\(a.k.a. Semi-implicit Euler Method)}
            & $\mathcal{O}(\Delta{t}^2)$ & $\mathcal{O}(\Delta{t})$ & $\mathcal{O}(\Delta{t})$ \\
            \hline
            \multlinecell{Midpoint Method\\(A Modified Euler Method)}
            & $\mathcal{O}(\Delta{t}^3)$ & $\mathcal{O}(\Delta{t}^2)$ & $\mathcal{O}(\Delta{t}^2)$ \\
            \hline
            \multlinecell{Verlet Integration\\(a.k.a. St\"{o}rmer--Verlet Method)}
            & $\mathcal{O}(\Delta{t}^4)$ & $\mathcal{O}(\Delta{t}^2)$ & $\mathcal{O}(\Delta{t})$ \\
            \hline
            Velocity Verlet & $\mathcal{O}(\Delta{t}^4)$ & $\mathcal{O}(\Delta{t}^2)$ & $\mathcal{O}(\Delta{t}^2)$ \\
            \hline
            Classic Runge--Kutta (a.k.a. RK4)
            & $\mathcal{O}(\Delta{t}^5)$ & $\mathcal{O}(\Delta{t}^4)$ & $\mathcal{O}(\Delta{t}^4)$ \\
            \hline
        \end{tabular}
    \end{table}

    The classic Runge--Kutta method provides the most accurate solution, but it exhibits a severe energy drift in
    large time intervals (since it is not a sympletic integrator).

    The velocity Verlet method performs the best in terms of conservation of energy. In terms of the displacement
    accuracy, it performs the same as the midpoint method and the regular verlet method.

    The Euler method can be unstable under certain conditions. Here, one example of this instability is evident.
    
    \fighere{sho-sol}{1}
    \fighere{sho-sol-euler}{1}
    \newgeometry{top=1in, bottom=1.5in}
    \fighere{sho-phase}{1}
    \fighere{sho-phase-euler}{1}

    \section{Logistic Map: Chaos}
    \fighere{bifurcation}{1}
    \fighere{bifurcation-zoom}{1}
    \restoregeometry
    \begin{empheq}[left={\text{Feigenbaum constants}\empheqlbrace}]{gather*}
        \text{first constant: } \delta = \frac{a_{n-1} - a_{n-2}}{a_n - a_{n-1}} \approx 4.669 \\
        \text{second constant: } \alpha = \frac{\text{width of a tine}}{\text{width of the next tine}} \approx 2.509
    \end{empheq}
    $\delta$ is calculated with $n=5$ and $\alpha$ is calculated with a branch of a carefully chosen 8 point tine.
    $\delta$ is accurate up to the third decimal place and $\alpha$ is accurate up to the second decimal place
    (the error is much larger for $\alpha$, as tine shapes are varied).

    I also made an animation (named \texttt{bifurcation.gif}) showing how the bifurcation diagram changes in each step
    of applying the logistic map.
\end{document}
